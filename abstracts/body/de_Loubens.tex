Magnetization dynamics is strongly nonlinear, yielding a series of interesting phenomena. It is well known that in ferromagnetic resonance of extended films, spin-wave (SW) instabilities quickly develop as the excitation power is increased \cite{Suhl_1957}, preventing to achieve large angle of uniform precession. The situation is quite different in nanostructures, where SW modes are highly quantized due to the geometric confinement, and expected to influence less the magnetization dynamics in the nonlinear regime. In this work, we probe the magnetization dynamics of an ultra-low damping YIG nano-disk \cite{Hahn_2014} using magnetic resonance force microscopy (MRFM). Ultra-large amplitude precession with complete suppression of the longitudinal component is achieved by pumping the sample with a uniform microwave field. Strikingly, we measure that the foldover shift is not constantly growing as the power is increased, but instead presents plateaus, pointing towards nonlinear energy dissipation to quantized SW modes, which is confirmed by micromagnetic simulations. In addition, by applying a second microwave field we are able to excite the SW resonances in the large-amplitude precession state. The lowest energy mode corresponds to the uniform nutation of the magnetization about its stable precession trajectory, whose frequency was shown to have a similar form as the Rabi formula, generalized to take into account nonlinearities \cite{Bertotti_2001}. Micromagnetic simulations show that higher order modes are nutation modes with spatial gradients, i.e., SW nutation modes.