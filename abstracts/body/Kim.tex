With the advent of cavity optomechanics, thermally-limited torque sensitivites are now accessible for low-mass on-chip torsional oscillators $-$ providing excellent test-beds for studying hybrid spin-oscillator systems, where spin dynamics are encoded in mechanical oscillations \cite{Kim_2013}. Here I will present our recent results to improve our high-performance nanoscale sensors by means of passive-cooling to reduce the thermal noise down to milli-Kelvin temperatures. Cooling the torsional mode at 25 mK, we achieved a remarkable torque sensitivity of 2.9 yNm/$\sqrt{\text{Hz}}$, just eleven times above the standard quantum limit for this device \cite{Kim_2016}. Furthermore, we explored a hybrid system with a magnetized iron-needle embedded on a torsional oscillator to drive and damp the torsional mode, where we can extract magnetic properties of the film \cite{Kim_2017}. Our on-chip device is sensitive, robust, and non-invasive, in which we can apply torque-mixing technique to unveil dynamic properties of spins and vortices \cite{Losby_2015}. We will report our current efforts to integrate in-plane and out-of-plane drive fields in a dilution refrigerator to study nanomagnetism and mesoscopic effects of superconductivity.