A skyrmion is a topologically protected spin texture, which is considered a potential candidate for future high-density storage devices. Most common are Bloch-type skyrmions (cf. Fig. 1 a), but also N\'{e}el-type skyrmions (cf. Fig. 1 b) exist, hosted e.g. in the lacunar spinels GaV\textsubscript{4}S\textsubscript{8} and GaV\textsubscript{4}Se\textsubscript{8} \cite{K_zsm_rki_2015}. They show a different orientation of the spin rotation, but also differ from their Bloch-type counterparts in that their orientation is determined by uniaxial anisotropy instead of an externally applied magnetic field. This anisotropy is apparent in a strong dependence of the magnetic phases on field orientation.
In order to quantify and understand the role of anisotropy in materials hosting N{\'e}el-type skyrmions, we perform dynamic cantilever magnetometry (DCM) \cite{Gross_2016} on single crystal samples of GaV\textsubscript{4}S\textsubscript{8} and GaV\textsubscript{4}Se\textsubscript{8}. In particular, we compare DCM results with a theoretical model of the magnetic phase diagram as a function of applied magnetic field magnitude and direction. By collecting magnetic torque signal for a series of applied field orientations, DCM reveals the magnetization, anisotropy, and magnetic phase diagram of our sample. Our results, which are in good agreement with the model, allow the extraction of a uniaxial anisotropy energy.