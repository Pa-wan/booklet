Beyond its primary objective of 3-dimensional, atomic-resolution imaging \cite{Nichol_2013}, magnetic resonance force microscopy (MRFM) technology also enables unprecedented measurement of other weak, small-scale magnetic phenomena in biology and physics. The technology enables single-copy characterization of the magnetic moment and hysteresis loop of nanoparticles \cite{Weber_2012}, the observation of spin diffusion in the solid state \cite{Cardellino_2014}, and the mapping of the magnetic fields resulting from nanoscale electromagnetic devices, allowing current densities \cite{Chang_2017} \cite{Yongsunthon_2001} and magnetic states \cite{Tao_2016} to be reconstructed.  Simulations suggest that AC currents ~100x lower than what NV center magnetometry can measure \cite{Chang_2017} can be observed using optimized nanofabricated probes.
To enable faster, more sensitive magnetic characterization across a wider range of sample temperatures, we have designed and built a new microscope for MRFM. Building upon work in the Rugar \cite{Z_ger_1993} \cite{Degen_2009} and Degen groups \cite{Moores_2015}, we have made improvements to manufacturing cost and ease of assembly, through simplified mechanical design and improved wire management. By incorporating a thermally insulating cantilever holder, heaters and thermometers, our microscope also allows independent control of scanned stage and cantilever temperatures, critical for examining biological samples at ambient temperatures.As a demonstration of the new microscope's capabilities, we have characterized the quality factor versus temperature of a single-crystal Si3N4 cantilever between 380K and 77K.