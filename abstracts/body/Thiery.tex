Generation and detection of pure spin currents circulating in magnetic materials through spin-orbit torque (SOT) has attracted recently a lot of attention, especially for future applications exploiting magnonic concepts. Among them, Yttrium Iron garnet (YIG), a magnetic insulator with very low damping, appears to be an excellent spin conductor \cite{Kajiwara_2010} since it can propagate very efficiently magnons (the spin carrier). It has been established \cite{Cornelissen_2015} that a pure spin current can be induced and detected between two Pt electrodes deposited few microns apart on top of YIG. Here we present a study of spin waves propagation in ultra-thin film of YIG (18nm) excited by large spin orbit torque. By injecting a high current density in Pt injector strip we are able to put our system strongly out of equilibrium. The main contribution of this work is an experimental evidence of a gradual spectral shift from thermal to subthermal magnon transport when a current density of $J_c= 5\cdot 10^{11}$ A$/$m$^2$ is injected in the Pt \cite{1702.05226v3}. This value corresponds to the expected threshold current for damping compensation of the Kittel mode. Our results suggest that a new spin conduction channel appear in the GHz frequency range when damping compensation is reached. This observation is supported by microfocus Brillouin light scattering measurement. We show that at such high current density, an exponential decrease of the electrical resistivity of the YIG must be taken into account and give rise to a parasitic non-magnetic offset voltage in the detector strip which is not related to magnon propagation \cite{1709.07207v1}.