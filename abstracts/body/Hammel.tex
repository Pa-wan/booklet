We and others \cite{Wolfe_2014,van_der_Sar_2015,Du_2017} have recently reported optical detection of magnetic resonance that avoids the need for resonant overlap of the target and nitrogen-vacancy (NV) sensor spins by exploiting target-spin induced NV spin relaxation. The relaxation is due to fluctuating magnetic fields which, in the case of ferromagnets (FM), result from the decay of the coherently driven mode into spinwaves. This provides a sensitive approach to measuring damping at the nanoscale.  Here we report NV-based MR spectroscopy of paramagnetic spins by a related mechanism.  As in the case of FMR detection, this is a broadband detection mode that is sensitive to excitation of spin dynamics at frequencies separate from the NV resonance conditions.  Here we describe spectroscopic measurements of P1 center spins in diamond, which reveal their expected anisotropic hyperfine structure and g-factor.   The technique can be extended to spins outside diamond, and should provide a versatile and sensitive technique for MR spectroscopy at the micro- to nanoscale.  This research is supported by the Army Research Office through Grant W911NF-16-1-0547, by the NSF MRSEC program through Grant DMR-1420451 and by the U.S. DOE through Grant DE-FG02-03ER46054.