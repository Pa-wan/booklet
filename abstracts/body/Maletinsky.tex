Electronic spins yield excellent quantum sensors, offering quantitative, nanoscale sensing and imaging\,\cite{Rondin_2014} down to the level of single spins\,\cite{Grinolds_2013}. Over the last years, the Basel Quantum Sensing Group has developed all-diamond scanning probes\,\cite{Appel_2016,Maletinsky_2012}, which offer robust, highly-sensitive platforms to employ Nitrogen-Vacancy (NV) centre spins for such single-spin quantum sensing, with a key scientific focus on condensed matter physics applications. I will describe our recent advances in applying this novel and unique quantum-sensing technology to study nano-magnetism at room temperature as well as mesoscopic systems in cryogenic environments down to the millikelvin range. 
Specifically, I will discuss the impact of NV magnetometry on the emerging field of antiferromagnetic spintronics \cite{Jungwirth_2016}, where our quantum sensors can address thin-film antiferromagnets with unprecedented performance to reveal nanoscale domains\,\cite{Kosub_2017} and non-trivial spin-textures \,\cite{Gross_2017}. In addition, I will present our recent advances in cryogenic NV magnetometry, including high-precision imaging of individual vortices in thin film superconductors \,\cite{Thiel_2016}, current-imaging in mesoscopic conductors and preliminary data from our recently completed millikelvin NV magnetometer. 
The performance and robustness of our nanoscale quantum sensing technology recently also led to the creation of ''Qnami'', the world's first company to commercialise nanoscale quantum sensing solutions. The scientific achievements of NV magnetometry demonstrated here and in several groups worldwide, now together with the commercial availability of user-friendly solutions for nanoscale quantum sensing, establish this technique as a powerful tool for magnetic sensing and imaging on the nanoscale $-$ a tool that we hope to see spreading into ever more laboratories in the coming years.