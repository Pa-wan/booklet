We investigate whether Magnetic resonance force microscopy (MRFM) is capable of maintaining its extremely high sensitivities and three-dimensional imaging with nanometer resolution when lowering its operation temperature by one or even two orders of magnitude compared to current work at temperatures between 0.5K and 4.2K. 
We present how we have successfully reduced cantilever heating during the read-out, by switching from laser interferometry to SQUID read-out and address its advantages and disadvantages. Additionally, we present our results so far, in generating the necessary B1 fields by sending a current through a superconducting microwire rather than a copper microwire. So far we have measured the nuclear spin-lattice relaxation times on copper at temperatures down to 42 mK, verified by the Korringa relation \cite{Wagenaar_2016} with an interaction volume of (30nm)$^3$. 
Despite the reduction in operation temperature and a record force sensitivity of less than 0.5 aN/$\sqrt{\text{Hz}}$ we find that this force sensitivity is severely reduced when approaching a surface. It turns out that cold electron spins on a surface lead to a large cantilever dissipation. We have conducted a detailed study of the dissipation and frequency shifts of a cantilever interacting with all surrounding spins, allowing us to experimentally determine the density and relaxation time of dangling bonds on a SiO2 surface \cite{de_Voogd_2017} and understanding some of the problems involving  two level systems. 
Finally, we have developed an innovative method for using the higher modes of the mechanical detector as radio-frequency (rf) source, removing the need for an on-chip rf source \cite{Wagenaar_2017}.