The limits of frequency resolution in nano NMR experiments have been discussed extensively in recent years. It is believed that there is a crucial difference between the ability to resolve a few frequencies and the precision
of estimating a single one. Whereas the efficiency of single frequency estimation gradually increases with the square root of the number of measurements, the ability to resolve two frequencies is limited by the specific time scale of the probe and cannot be compensated for by extra measurements. In this talk I will show  that the relationship between these quantities is more subtle and both are only limited by the Cramer-Rao bound of a single frequency estimation. The talk is based on \cite{Schmitt_2017,1707.01902v1}.