Spin angular momentum which is one of the degrees of freedom in electron can be mutually converted with a macroscopic rotation according to the conservation low of angular momentum. The conversion from the spin angular momentum to the macroscopic rotation was experimentally demonstrated in ferromagnetic bodies by Einstein and De Haas while Barnett succeeded its inverse conversion. Very recently, from the analytical solution of the Dirac equation with a general covariance, Matsuo et al. theoretically predicted that the same kind of mutual conversion can be realized for free electrons in non-magnetic metals with a weak spin orbital coupling \cite{Matsuo_2013}. We demonstrated a conversion of alternating spin current (SC) from a macroscopic rotation generated by surface acoustic wave (SAW) which propagates in a NiFe / Cu bilayer deposited on a LiNbO$_3$ substrate \cite{Kobayashi_2017} \. An FMR excited in the NiFe layer was successfully observed when the fundamental frequency of SAW matched with the FMR frequency. The strength of FMR excitation was strongly suppressed when the Cu layer was removed from the bilayer or an insulating SiO$_2$ layer was inserted in the interface of the bilayer. This is the clear evidence that the alternating SC generated in Cu layer via spin-rotation coupling (SRC) plays an important role for the FMR excitation. The angular dependence of the strength of FMR excitation quantitatively supports the successful generation of alternating SC using SAW via SRC. Our experimental result will open the way to generate an alternating SC in variety of SAW devices without using ferromagnets and/or nonmagnetic materials with large spin-orbital coupling.