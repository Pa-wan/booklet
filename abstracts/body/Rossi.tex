Self-assembled nanowire (NW) crystals can be grown into nearly defect-free mechanical resonators with small motional mass, high resonant frequency and low dissipation. Besides being extremely sensitive force sensors, their highly symmetric cross-section results in flexural modes doublets that are nearly degenerate and allow to detect forces along two orthogonal directions. Such NWs enable a form of force microscopy capable of mapping both size and direction of tip-sample forces \cite{Rossi_2016,de_L_pinay_2016}.
	In order to extend these capabilities to magnetic force microscopy, here we investigate GaAs NWs $-$ 200 nm in diameter $-$ grown by molecular beam epitaxy to have a MnAs ferromagnetic crystal at their tip \cite{Hubmann_2016}. By monitoring the mechanical resonance frequencies of the NW's mode doublet as a function of the applied field \cite{Gross_2016}, it is possible to reveal the saturation magnetization, anisotropy energy, and reversal of the magnetic tip. Moreover, additional details on the magnetic anisotropy axis can be accessed thanks to the vectorial nature of the NW sensor. By comparison with numerical simulations, we identify signatures in the magnetic torque related to magnetization switching and even to the nucleation of a magnetic vortex. Using a well-known magnetic field profile, generated by a current carrying wire, we also determine the tip-equivalent magnetic surface charge density to calibrate the sensor's response to inhomogeneous magnetic field.
	In principle, such NW sensors, due to their nanometric magnetic tip and their high force sensitivity, are capable of mapping stray magnetic fields with enhanced sensitivity and resolution compared to the state of the art.