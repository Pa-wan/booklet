Magnetic resonance force microscopy (MRFM) allows investigation of various kinds of spin-related material properties in small sample volumes. We reduced the operating temperature of this technique by 2 orders of magnitude to 10 mK. As a  demonstration, we measured the nuclear spin-lattice relaxation time on copper at temperatures down to 42 mK, verified by the Korringa relation \cite{Wagenaar_2016} with an interaction volume of $(30 \text{nm})^3$. 
Furthermore, we have conducted a detailed study of the dissipation and frequency shifts of a cantilever interacting with all surrounding spins, allowing us to experimentally determine the density and relaxation time of dangling bonds on a $\text{SiO}_2$ surface \cite{de_Voogd_2017} and impurity spins in bulk diamond. This enables us to understand some of the problems involving 2LS, one of the bottlenecks in the development of optomechanical-like hybrid quantum devices. 
Finally, we have developed an innovative method for using the higher modes of the mechanical cantilever as radio-frequency (rf) source, removing the need for an on-chip rf source \cite{Wagenaar_2017}. This should be considered an important step towards an MRFM which can be widely used in condensed matter physics, for instance to investigate inhomogeneous electron systems.