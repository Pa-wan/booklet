Numerous techniques for the detection of NMR signals with high resolution in both the spectral and the spatial domain have been proposed. Among these, inductive detection by miniaturized coils \cite{Olson_1995} and nanoscale magnetometry using nitrogen-vacancy defects in diamond \cite{Wrachtrup_2016} facilitate detection under ambient conditions.
Here, giant magneto-resistive (GMR) sensors are assessed for NMR detection at ambient conditions. The aim for GMR detection is the picotesla field sensitivity at radio frequency combined with the small sensor size in the micrometer regime \cite{Freitas_2007,Fermon_2013}. As shown recently in our lab, a GMR sensor immersed into water or ethanol can detect the $^1$H NMR spectrum of picoliter volumes at 0.3 T / 13 MHz \cite{Guitard_2016}. In this way, the GMR sensor detects the magnetic fields due to coherent precession of close-by nuclear spins.
In this contribution, we report on experiments at 0.7 T / 30 MHz and outline key aspects of GMR-detected NMR. Particular emphasis is laid on the influence of the 0.7 T field on the GMR sensor, which enforces an experimental alignment procedure. Using a sensor of 12x100 $\mu$m$^2$, the spectrum of water doped by a relaxation agent is obtained with an SNR of 10 and a linewidth of 1.4 ppm after 30 minutes. These figures are the basis for our ongoing optimization of the NMR performance. First, we consider potential susceptibility effects, which could arise due to the ferromagnetic layers inside the GMR sensors. Second, we indicate how further sensitivity enhancement could be achieved with sensors based on magnetic tunnel junctions.