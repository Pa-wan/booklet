Cavity Spintronics (also known as Spin Cavitronics) is a newly developing interdisciplinary field that brings together microwave cavity community with researchers from spintronics. This field started around 2014 when it was found that ferromagnets in cavities hybridize with both microwaves and light via light-matter interaction. Since then, the emergence of this field has attracted broad interests. At the center stage of the topic is the physics of a quasi-particle called cavity magnon polariton (CMP). Via the quantum physics of spin-photon entanglement on the one hand, and via classical electrodynamic coupling on the other, CMP connects some of the most exciting modern physics, such as quantum information and quantum optics, with one of the oldest science on the earth, the magnetism. 

In this new community, most groups utilize the hybrid and nonlocal nature of CMP to develop cavity-mediated coupling techniques for both spintronic and quantum applications. This stream of research, including our recent demonstration of cavity-mediated distant control of spin current \cite{Bai_2017}, root on single-particle physics of CMP. In this talk, we will present a distinct feedback-coupled cavity technique, which we develop to study the cooperative dynamics of trillions of CMP. Utilizing such coherent dynamics of CMP ensembles, we demonstrate the control of magnon-photon Rabi frequency by changing the photon Fock state occupation, and we discover the evolution of CMP to cavity magnon triplet and cavity magnon quintuplet [Nature Comm. (2017), to be published]. Our results may open up new avenues for exploiting the light-matter interactions.