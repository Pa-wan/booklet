A precessing ferromagnetic needle magnetometer is proposed in \cite{Jackson_Kimball_2016}, comparing with the alkali vapor cell atomic magnetometer. The needle magnetometer has a better fundamental sensitivity due to much higher density and the atoms precess coherently. Based on the theoretical estimate, the needle magnetometer's sensitivity could surpass the standard quantum limit (SQL). The purpose of levitating the ferromagnetic particle is letting the particle freely precess. The Meissner effect of the superconductors is promising for levitation. The flux pinning inside the type-I superconductor (lead) prevents the micro-particle from stable levitation, when the size of the particle is comparable with size of the flux pinning fringe of lead, the micro-particle could be easily attracted by the flux pinning and fall to the surface. The type-II superconductors are usually in mixed state, however, when the type-II superconductor has a large Hc1, and the external magnetic field is below Hc1, then Meissner effect is dominant. Another requirement for observing the needle particle processing about the magnetic field instead of orienting to the magnetic field is the angular momentum of total spins is much larger than the mechanical orbital angular momentum. In this case, the external magnetic field needs to be shielded/compensated. For a 10 um length needle, the field needs to be smaller than 1 nT. Another advantage of the superconducting levitation is that the bowl-shaped superconductor also acts as superconducting magnetic shield. This research could tell the boundary between the classical physics and the quantum physics.