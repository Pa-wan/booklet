Magnetic droplet solitons were first predicted to occur in materials with uniaxial magnetic anisotropy due to a long-range attractive interaction between elementary magnetic excitations,magnons. A non-equilibrium magnon population provided by a spin-polarized current in nanocontacts enables their creation and there is now clear experimental evidence for their formation including direct images obtained with scanning x-ray transmission microscopy. Interest in magnetic droplets is associated with their unique magnetic dynamics
that can lead to new types of high frequency nanometer scale oscillators of interest for information processing, including in neuromorphic computing. However, there are no direct measurements of the time required to nucleate droplet solitons or their lifetime—experiments to date only probe their steady-state characteristics, their response to dc spin-currents. Here we show in experiments with short current pulses that there are very different timescales for
droplet annihilation and generation: annihilation occurs in a few nanoseconds while generation can take up to a microsecond. Micromagnetic simulations illustrate the magnetization dynamics associated with these processes and show that there can be an incubation delay for droplet generation.