One of the grand challenges in quantum and condensed matter physics is to harness the spin of electrons for information technologies. While spintronics, based on charge transport by spin polarized electrons, made its leap in data storage by providing extremely sensitive detectors in magnetic hard-drives, it turned out to be challenging to transport spin information without great losses. With magnonics a visionary concept inspired researchers worldwide: Utilize magnons - the collective excitation quanta of the spin system in magnetically ordered materials - as carriers for spin information.

While macroscopic prototypes of magnonic logic gates already have been demonstrated, the full potential of magnonics lies in the combination of magnons with nano-sized spin textures. Both magnons and spin textures share a common ground set by the interplay of dipolar, spin-orbit and exchange energies rendering them perfect interaction partners. Magnons are fast, sensitive to the spins' directions and easily driven far from equilibrium. Spin textures are robust, non-volatile and still reprogrammable on ultrashort timescales.

In this presentation I will discuss magnon propagation in spin textures and how these magnons can be locally excited in spin textures by pure spin currents \cite{Wagner_2016,Vogt_2014}.