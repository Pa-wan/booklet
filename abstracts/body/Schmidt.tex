Nano electromechanical systems have seen huge progress over the last decades. Recently coupling of mechanical to electromagnetic modes has gained importance because of possible use in so called transmons and their application for quantum computing \cite{O_Connell_2010},\cite{Chu_2017} . Yttrium Iron Garnet would be extremely interesting as a nanoresonator material because it combines long lifetimes as well for spin waves (magnons) as for mechanical waves (phonons) and a coupling mechanism for both (magnetoelastic coupling or magnetostriction). Coupling of magnons to phonons in YIG spheres of diameters of a few hundred micron has already been demonstrated\cite{Zhang_2016} but making true 3D YIG nanoresonators would open up a new field of applications for nanooscillators. However, up to now no method was available to shape three dimensional nanostructures from monocrystalline YIG. We have developed a process to build monocrystalline freestanding 3D YIG nanoresonators. The structures can be designed as suspended bridges, cantilevers but also as more complex structures like for example suspended rings or disks. The structures were investigated using transmission electron microscopy indicating high crystalline quality. In ferromagnetic resonance different modes can be observed. The linewidth for an ensemble of 8000 bridges was as small as 
7 Oe@6.5 GHz. Further investigation was done using time and spatially resolved Kerr microscopy. Here we see various standing spin waves including Damon Eshbach Modes and Backward Volume Modes. Based on these measurements also the damping for a single resonator could be determined to $\alpha < 4\times 10^{-4}$. The modes observed can be nicely reproduced in 3D micromagnetic simulations.