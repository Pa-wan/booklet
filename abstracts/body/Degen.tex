Nanoscale magnetic resonance imaging (NanoMRI) is a challenging endeavor with important potential applications in nanomedicine, chemistry and solid state physics. One of the most advanced technologies in the field is magnetic resonance force microscopy (MRFM), which combines force-detected scanning microscopy and spin control through nuclear magnetic resonance. Progress in this field must often be earned through paradigm-shifting new technologies. In this talk, we will present a number of such innovations at various stages of implementation, ranging from the use of commercial hard drive heads as field sources, to novel mechanical sensor geometries, to surprising sensing methods using parametric resonators.