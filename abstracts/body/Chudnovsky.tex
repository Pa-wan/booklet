Presence of topological defects in spin systems with quenched randomness has a profound effect on the long-range behavior \cite{Proctor_2014,Proctor_2015}. Application of a weak magnetic field to a 2D film with random local anisotropy results in a skyrmion glass, FIG.1. Quenched randomness stabilizes skyrmions that would otherwise collapse due the violation of the scale invariance by the atomic lattice \cite{Cai_2012}. Good understanding of the properties of the skyrmion glass, such as field dependence of the concentration of skyrmions, their average size and stability, has been achieved via scaling arguments and confirmed by large-scale numerical studies of spin lattices \cite{1710.10608v1}. When quenched disorder is weak, evolution of labyrinth domains into compact topological structures on application of the magnetic field is governed by random configuration of Bloch lines inside domain walls \cite{1706.02994v1}. Depending on the combination of Bloch lines, the magnetic domains evolve into individual skyrmions, biskyrmions, or more complex topological objects. While the geometry of such objects is sensitive to the parameters, their topological charge is uniquely determined by the topological charge of Bloch lines inside the magnetic domain from which the object emerges.